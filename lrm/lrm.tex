\documentclass[12pt]{article}
\usepackage[utf8]{inputenc}
\usepackage[dvipsnames]{xcolor}
\usepackage{hyperref}
\hypersetup{
    colorlinks=true,
    linkcolor=blue,
    }
\usepackage[margin=1.0in]{geometry}
\usepackage{listings}
\usepackage{graphicx}
\graphicspath{ {./} }

%% Frames setup
\usepackage[framemethod=TikZ]{mdframed}
\mdfsetup{skipabove=\topskip,skipbelow=\topskip}

%% Configure fonts
\usepackage[default,scale=0.95]{opensans}
\usepackage{inconsolata}
\usepackage[T1]{fontenc}

\begin{document}

\definecolor{LightGray}{gray}{0.6}

% +-------------------------------+
% | Define HeBGB listing language |
% +-------------------------------+
\lstdefinelanguage{HeBGB}{
    basicstyle=\ttfamily,
    columns=fullflexible,
    sensitive=false, % keywords are not case-sensitive  
    keywords = {},
    otherkeywords={ % punctuation marks
        <-, <~, ?, :
    },
    keywordstyle=\color{Mahogany}\bfseries,
    morekeywords=[2]{ % types
        int, list, string,
    },
    keywordstyle=[2]\color{YellowGreen},
    morekeywords=[3]{
        in, continue, with, then
    },
    keywordstyle=[3]\color{Mahogany}\bfseries,
    morecomment=[l]{;;},
    commentstyle=\color{LightGray}\ttfamily\textit,
    morestring=[b]", % strings are enclosed in double quotes
    stringstyle=\color{OliveGreen}\ttfamily,
    showspaces=false,
    showstringspaces=false
} % 

% +----------------------------------------+
% | Define production rule custom commands |
% +----------------------------------------+
\newcommand{\Comment}[1]{\color{LightGray}#1}
\newcommand{\Term}[1]{\textbf{\color{Mahogany}\texttt{#1}}}
\newcommand{\Non}[1]{\textit{#1}}
\newcommand{\Eq}{$::=$}
\newcommand{\Or}{$|$}

% +-------+
% | Title |
% +-------+
\begin{center}
\includegraphics[width=2cm]{logo.png}

\bigskip
\begin{bfseries}
\begin{Large}The HeBGB Programming Language\end{Large}
\end{bfseries}

\bigskip
Craig Cagner, Sam Cohen, James Eidson, and Paul Roche

\bigskip
Spring, 2022
\end{center}

% +---------------------------------+
% | Create the rule box environment |
% +---------------------------------+
%% Make a new counter to keep track of the current
%% grammar rule
\newcounter{rule}[section]\setcounter{rule}{0}
\renewcommand{\therule}{\arabic{section}.\arabic{rule}}

\newenvironment{rule_box}[1][]{
\refstepcounter{rule}
\ifstrempty{#1}%
% if condition (without title)
{\mdfsetup{%
frametitle={%
    \tikz[baseline=(current bounding box.east),outer sep=0pt]
    \node[anchor=east,rectangle,fill=white]
    {\strut \therule};}
}%
% else condition (with title)
}{\mdfsetup{%
frametitle={%
    \tikz[baseline=(current bounding box.east),outer sep=0pt]
    \node[anchor=east,rectangle,fill=white]
    {\strut \therule:~#1};}%
}%
}%
% Both conditions
\mdfsetup{%
    innertopmargin=10pt,linecolor=black,%
    linewidth=1pt,topline=true,%
    frametitleaboveskip=\dimexpr-\ht\strutbox\relax%
}
\begin{mdframed}[]\relax}{%
\end{mdframed}}

% +---------------------------------+
% | Create the code box environment |
% +---------------------------------+
%% Make a new counter to keep track of the current
%% code box
\newcounter{codebox}[subsection]\setcounter{codebox}{0}
\renewcommand{\thecodebox}{\arabic{section}.\arabic{subsection}.\arabic{codebox}}

\newenvironment{code_box}{
\refstepcounter{codebox}
\mdfsetup{%
frametitle={%
    \tikz[baseline=(current bounding box.east),outer sep=0pt]
    \node[anchor=east,rectangle,fill=white]
    {\strut \thecodebox};}
}%
% Both conditions
\mdfsetup{%
    innertopmargin=10pt,linecolor=black,%
    linewidth=1pt,topline=true,%
    frametitleaboveskip=\dimexpr-\ht\strutbox\relax%
}
\begin{mdframed}[]\relax}{%
\end{mdframed}}

% +------------------------------+
% | Document Content Begins Here |
% +------------------------------+
\section{Introduction}
HeBGB is a programming language centered around first-class functions
and first-class continuations. By exposing continuations to the user,
HeBGB allows complex control structures such as iterators, co-routines,
and exceptions to be implemented by user code.

HeBGB programs are composed of expressions, and most bindings are immutable.
HeBGB encourages a functional programming style, though variable assignment
is possible and necessary to use some of the language features.
The language has a small core of primitive functions and operations, which
are meant to be used to build custom tools that are targeted towards
the programmers' exact problem domain.
\section{Lexical Elements}
HeBGB programs are utf8-encoded text files. Every token in a valid
HeBGB program is either a keyword, an identifier, punctuation, or a value
literal. \ref{keywords} shows the 8 keywords in the language. 
HeBGB programs contain lots of identifiers. Identifiers 
are strings that aren't a keyword and don't contain 
\Term{"}, \Term{(}, \Term{)}, \Term{\{}, \Term{\}}, \Term{[}, or \Term{]}, 
and are not valid representations of integer literals. 
HeBGB's syntax also includes numerous punctuation marks. These
can also be considered keywords in the language. Just like keywords,
identifiers can \textit{contain} them, but cannot be \textit{equal} to them.

\begin{rule_box}[Keywords]
\label{keywords}
\Term{in} \qquad \Term{continue} \qquad \Term{with} \qquad \Term{then} \qquad
\Term{int} \qquad \Term{string} \qquad \Term{cont} \qquad \Term{list}
\end{rule_box}

There are 4 types of value literals in HeBGB. There are integer literals,
list literals, tuple literals, and string literals. Integer literals are
a sequence of decimal characters (\Term{0}-\Term{9}) optionally prefixed
by a negation mark (\Term{-}). List and Tuple literals are described
in \ref{productions:helpers}. String literals are written as a \Term{"},
followed by a list of characters, followed by a closing \Term{"}.

\section{Formal Grammar}

\begin{rule_box}[Expressions]

\begin{tabular}{r c l r}
\Non{exp} & \Eq & \Non{name} \\
          & \Or & \Non{const} \\
          & \Or & \Term{(} \Non{exp} \Term{)} & \Comment{grouping} \\
          & \Or & \Non{exp} \Term{(} \Non{args} \Term{)} & \Comment{$\lambda$ application} \\
          & \Or & \Non{exp} \Term{()} \\
          & \Or & \Non{ty} \Non{name} \Term{<-} \Non{exp} \Term{in} \Non{exp} & 
          \Comment{let binding}\\
          & \Or & \Non{name} \Term{<\char`~} \Non{exp} \Term{in} \Non{exp} & 
          \Comment{variable assignment}\\
          & \Or & \Non{ty} \Non{name} \Term{(} \Non{params} \Term{) ->} 
          \Non{exp} \Term{in} \Non{exp} & 
          \Comment{$\lambda$ binding}\\
          & \Or & \Non{ty} \Non{name} \Term{() ->} 
          \Non{exp} \Term{in} \Non{exp} \\
          & \Or & \Non{ty} \Non{name} \Term{<- continue with} \Non{name} \Term{in} \Non{exp} 
          \Term{then} \Non{exp} & \Comment{continuation} \\
          & \Or & \Term{if} \Non{exp} \Term{do} \Non{exp} \Term{else} \Non{exp} & \Comment{conditional}\\
          & \Or & \Non{infix}\\\\
\Non{infix} & \Eq & \Non{exp} \Term{+} \Non{exp}
            ~\Or~ \Non{exp} \Term{-} \Non{exp}
            ~\Or~ \Non{exp} \Term{*} \Non{exp} \\
            & \Or & \Non{exp} \Term{/} \Non{exp}
            ~\Or~ \Non{exp} \Term{\%} \Non{exp}
            ~\Or~ \Non{exp} \Term{\{} \Non{exp} \\
            & \Or & \Non{exp} \Term{\}} \Non{exp}
            ~\Or~ \Non{exp} \Term{<=} \Non{exp}
            ~\Or~ \Non{exp} \Term{>=} \Non{exp} \\
            & \Or & \Non{exp} \Term{=} \Non{exp}
            ~\Or~ \Non{exp} \Term{!=} \Non{exp}
            ~\Or~ \Non{exp} \Term{||} \Non{exp} \\
            & \Or & \Non{exp} \Term{\&\&} \Non{exp}
\end{tabular}
\end{rule_box}
\begin{rule_box}[Types]
\begin{tabular}{r c l r}
\Non{ty} & \Eq & \Term{int} \\
         & \Or & \Term{string} \\
         & \Or & \Term {cont} & \Comment{The type of continuation values}\\
         & \Or & \Non{ty} \Term{list} & \Comment{The type of lists}\\
         & \Or & \Non{ty} \Term{*} \Non{ty} & \Comment{The type of tuples}\\
         & \Or & \Non{ty} \Term{->} \Non{ty} & \Comment{The type of functions}
\end{tabular}
\end{rule_box}
\begin{rule_box}[Helpers]
\label{productions:helpers}
\begin{tabular}{r c l r}
\Non{const} & \Eq & \Non{number} \\
            & \Or & \Term{[} \Non{args} \Term{]} \\
            & \Or & \Term{\{} \Non{args} \Term{\}} \\
            & \Or & \Term{"}\Non{string}\Term{"} \\
\end{tabular}
\begin{tabular}{r c l r}
\Non{args} & \Eq & \Non{exp} \\
            & \Or & \Non{exp} \Term{,} \Non{args} \\
\end{tabular}
\begin{tabular}{r c l r}
\Non{params} & \Eq & \Non{ty} \Non{exp} \\
             & \Or & \Non{ty} \Non{exp} \Term{,} \Non{params} \\
\end{tabular}
\end{rule_box}
\section{Types and Values}
Every HeBGB program has just 6 kinds of values. There are integers (\Term{int}),
strings (\Term{string}), continuation values (\Term{cont}), 
lists (\Non{a} \Term{list}), tuples (\Non{a} \Term{*} \Non{b}), 
and function types (\Non{a} \Term{->} \Non{b}). Lists, tuples, and
function types may be \textit{composed} to yield more complicated types.
The types of the language closely resemble the values that they represent.
To this end, HeBGB's type system is structural; types are considered compatible
if their algebraic representations are equal, and there is no way to
name a type in HeBGB.
\section{Expression Semantics}
\subsection{Constants}
HeBGB has four types of constants: integers, lists, tuples, and strings.
\ref{code:constants} shows the syntax for how to write all four. On each
line, the \Term{<-} operator is used to bind a name to the value of each
expression on the right-hand side. Also, note that before each name, it is 
necessary to name the type of the value being bound. The compiler will 
ensure that the types mentioned are consistent with the other expressions 
in the program.
\begin{code_box}
\label{code:constants}
\begin{lstlisting}[language=HeBGB]
;; The integer 42 is bound to the name myint
int myint <- 42 in
;; We can also bind lists
int list ints <- [1, 2, 3] in
;; And tuples...
int * int origin <- {0, 0} in
;; And even tuples of lists...
int list * int list nums <- {[0, 2], [1, 3]} in
string message <- "Hello, world" in 0
\end{lstlisting}
\end{code_box}
A HeBGB program can be seen as one big expression, the value of which
evaluates to the exit code of the program. In the final line of
\ref{code:constants} we find the constant \texttt{0} which is the
exit code of the program.
\subsection{Names and Bindings}
In HeBGB, most expressions must be bound to a name. Names are
bound to values with the \Term{<-} operator. It is also possible to bind
a new value to a previously bound name with the \Term{<\char`~} operator.
Note that re-binding a name with the \Term{<\char`~} operator is a
true assignment; future accesses of the name will yield the \textit{new}
value of the name.

Names bound by \Term{<-} are visible to every sub-expression of the
binding after the \Term{in} keyword. 
\ref{code:scope} gives an example of how names can be
scoped to sub-expressions within a program. In the example,
the definition of \texttt{e} has \texttt{a} and \texttt{b} in scope,
but \texttt{c} and \texttt{d} are not in scope, because they are
bound only in the definition of \texttt{b}.

The HeBGB compiler can determine at compile-time which names will
be in scope for any expression in the program. Programs that mention
names that are not in scope will be rejected.
\begin{code_box}
\label{code:scope}
\begin{lstlisting}[language=HeBGB]
int a <- 2 in
int b <- 
  int c <- 42 in    ;; a is in scope
  int d <- 0  in    ;; a and c are in scope
  c + d in          ;; a, c, and d are in scope
int e <- 3 in a + b ;; a and b are in scope
\end{lstlisting}
\end{code_box}
\subsection{Conditional Expressions}
The conditional expression is used to choose an expression to be
evaluated at run-time. The conditional expression has three sub-expressions:
the test, the true branch, and the false branch. When evaluated, the
if expression evaluates the test expression. If the test expression yields
the integer value \texttt{0}, the false branch is evaluated. Any other
value will result in the true branch being evaluated.

Example \ref{code:conditional} shows an expression which clamps
integers between 0 and $\infty$. In the test of the conditional,
a comparison operator is used to determine if the value bound to
\texttt{a} is less than 0. If it is, \texttt{0} is returned. Else,
\texttt{a} must be equal to or greater than 0, so its value is
returned unchanged.
\begin{code_box}
\label{code:conditional}
\begin{lstlisting}[language=HeBGB]
int a <- -10 in
int pos_a <- if a < 0 do 0 else a in
pos_a ;; pos_a is 0
\end{lstlisting}
\end{code_box}
The test, true branch, and false branch sub-expressions of a conditional
can contain arbitrary expressions, including comparisons, bindings, and
other conditionals. Example \ref{code:conditional2} is another 
clamping expression, except this time \texttt{a} is clamped between 
0 and 100. In \ref{code:conditional2}, we use nested conditional expressions
to perform checks on the upper and lower bound of the clamp. 

\begin{code_box}
\label{code:conditional2}
\begin{lstlisting}[language=HeBGB]
int a <- 20 in
int clmp_a <- if a < 0 do 0 else
              if a > 100 do 100 else a
clmp_a ;; clmp_a is 20
\end{lstlisting}
\end{code_box}

\subsection{Function Definition and Application}
Functions are one of the central features of HeBGB. In HeBGB, functions
are first-class values that can be passed around to other functions
and bound to names. In \ref{code:succ}, a function is defined that
accepts an integer and returns its successor.
\begin{code_box}
\label{code:succ}
\begin{lstlisting}[language=HeBGB]
int succ(int n) -> n + 1 in
int num <- 0 in
succ(num)
\end{lstlisting}
\end{code_box}
The first line of \ref{code:succ} defines a function, \texttt{succ},
which accepts a single argument of type \texttt{int}. The body
of the function is a single expression that is evaluated when the
function is called. Note that \texttt{succ} is a value like any other.
As such, we can bind it to a new name and call \textit{that} function
instead.
\begin{code_box}
\label{code:add-one}
\begin{lstlisting}[language=HeBGB]
int succ(int n) -> n + 1 in
int -> int add-one <- succ in
int num <- 0 in
add-one(num)
\end{lstlisting}
\end{code_box}
\ref{code:add-one} behaves identically to \ref{code:succ}. On the
second line of \ref{code:add-one}, we have re-bound \texttt{succ} to
\texttt{add-one}, copying the underlying function value. Note that
since \texttt{succ} is still bound, we have created an environment where
\texttt{succ} and \texttt{add-one} are effectively aliases to the same
function.

On line 2 of \ref{code:add-one} we are required to name the type
of \texttt{add-one} when it is bound to \texttt{succ}. The types
of functions are written \texttt{a -> b} where \texttt{a} is the
type of the first argument and \texttt{b} is the return type. If a
function takes multiple arguments, more arrows are used between each
argument type. For example, the type of a function that takes a
string and two ints and returns a string list is
\texttt{string -> int -> int -> string list}.

Function application is performed by writing an argument list next
to an expression. Argument lists are comma-separated lists of expressions
surrounded by parenthesis. In \ref{code:succ}, \texttt{succ} is applied
with \texttt{num} by the expression \texttt{succ(num)}. Functions need
not be simple identifiers, however.

Say we wanted to write a function that could give us the nth successor
to a number, not just the 1st successor. We could write a function that
takes two arguments, but in HeBGB, we can also write a function that
returns another function. \ref{code:nsucc} gives such an example.
\begin{code_box}
\label{code:nsucc}
\begin{lstlisting}[language=HeBGB]
int -> int nsucc(int n) ->
  int succ(int p) -> n + p in
  succ in
int -> int five_succ <- nsucc(5) in
five_succ(10) ;; is 15
\end{lstlisting}
\end{code_box}
let's break down example \ref{code:nsucc}:\\
\begin{tabular}{p{0.5\linewidth} p{0.4\linewidth}}
\begin{lstlisting}[language=HeBGB, escapechar=!]
int -> int nsucc(int n) ->
  !\color{LightGray}{int succ(int p) -> n + p in}!
  !\color{LightGray}{succ}! in
!\color{LightGray}{int -> int five\_succ <- nsucc(5) in}!
!\color{LightGray}{five\_succ(10)}!
\end{lstlisting} & 
We define a function, \texttt{nsucc} which takes an integer and
returns a function of type \texttt{int -> int}.
\\
\end{tabular}

\noindent
\begin{tabular}{p{0.5\linewidth} p{0.4\linewidth}}
\begin{lstlisting}[language=HeBGB, escapechar=!]
!\color{LightGray}{int -> int nsucc(int n) ->}!
  int succ(int p) -> n + p in
  succ in
!\color{LightGray}{int -> int five\_succ <- nsucc(5) in}!
!\color{LightGray}{five\_succ(10)}!
\end{lstlisting} & 
The body of \texttt{nsucc} is itself a function
which adds its argument, \texttt{p}, to \texttt{n}. Since \texttt{n} 
is \textit{captured} by \texttt{succ}, it is available to the body
of \texttt{succ} even when \texttt{nsucc} is no longer in scope.
\\
\end{tabular}

\noindent
\begin{tabular}{p{0.5\linewidth} p{0.4\linewidth}}
\begin{lstlisting}[language=HeBGB, escapechar=!]
!\color{LightGray}{int -> int nsucc(int n) ->}!
  !\color{LightGray}{int succ(int p) -> n + p in}!
  !\color{LightGray}{succ in}!
int -> int five_succ <- nsucc(5) in
five_succ(10)
\end{lstlisting} & 
\texttt{five\_succ} binds a closure of \texttt{succ} where \texttt{n} is
equal to 5. Applying \texttt{five\_succ} yields 15.
\\
\end{tabular}
\subsection{Continuations}
Continuations in HeBGB are created with the \Term{continue with} keyword.
\Term{continue with} is followed by a \texttt{name}, a handler body,
$\texttt{e}_1$, and a following expression, $\texttt{e}_2$.
When \Term{continue with} is evaluated, a continuation representing the
current environment is captured and bound to \texttt{name}. Immediately
after capture, $\texttt{e}_1$ is evaluated with \texttt{name} bound to
the continuation's value. The continuation value behaves as a function to
the user's code; continuation values can only be applied, and they only
accept applications of a single argument. Once $\texttt{e}_1$ has been
evaluated, $\texttt{e}_2$ is evaluated in the same manner as other bindings 
in the language.

If the continuation value bound to \texttt{name} is applied at any time
in the program's execution following its introduction, 
the context captured by the continuation is restored, and the value 
of the argument passed to the application is returned from the 
\Term{continue with} expression that generated the continuation value.

Example \ref{code:safe-divide} shows how continuations
can be used to alter the control flow of a HeBGB program.
\begin{code_box}
\label{code:safe-divide}
\begin{lstlisting}[language=HeBGB]
int safe-divide(int a, int b) ->
  int result <- continue with k in
    a / (if zero(b) do k(-1) else b)
  then result in
safe-divide(10, 0) ;; returns -1
\end{lstlisting}
\end{code_box}
\ref{code:safe-divide} implements a function, \texttt{safe-divide},
which avoids divide-by-zero errors by returning \texttt{-1} when
a zero divisor is detected. This example emulates a basic form
of return semantics from imperative languages where after return
has been called (roughly equivalent to \texttt{k(-1)} in this
example), no more calculations in the body of the function are
attempted, and the result is ``escaped'' immediately from the
surrounding code.
\end{document}
